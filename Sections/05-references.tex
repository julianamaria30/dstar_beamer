%----------------------------------------------------SLIDE------------------
 \begin{frame}[t, allowframebreaks]{References}
 %\frametitle{References}
%\begin{frame}{Reference}
    %\transboxin[duration=1,direction=30]

    % \begin{bibunit}[plain]
    % \cite{guangyi2018research}.
    % %\cite{kanakia2012}
    % %\cite{agostini2007}
    % %\cite{azuma1997survey}
    % \cite{Buss2005}
  
    % \putbib
    % \end{bibunit}
  
    %\bibliographystyle{IEEEtran}
    %\bibliographystyle{IEEEtranS}
    %\bibliographystyle{IEEEbib}
    \bibliographystyle{abntex2-alf}
    %\bibliographystyle{abntex2-num}
    %\bibliographystyle{abnt-alf}
    \bibliography{bibliography} 
    %\putbib

%*----------- notes
    %\note[item]{Notes can help you to remember important information. Turn on the notes option.}
\end{frame}
%

 %*----------- SLIDE (ANDAMENTO DO PROJETO) -------------------------------------------------------------
 \begin{frame}[t]{Conceitual} 
    \framesubtitle{Objetivo}
    \begin{table}[ht!]
    \centering
        \caption{PERCENTUAL DE CONCLUSÃO}
        \begin{tabular}{|c|c|c|c|c|} \hline
            \textbf{Estado Anterior}&\textbf{Estado Atual}&\textbf{Eficiência}\\\hline
            75\% &81\% &85.3\% \\ \hline
        \end{tabular}
    \end{table}
    \begin{table}[ht!]
        %\centering
            \caption{ATIVIDADES REALIZADAS}
            \begin{tabular}{|c|c|c|c|} \hline
                \textbf{Atividade}&\textbf{Porcent.}&\textbf{Atividade}&\textbf{Porcent.}\\\hline
                Simulação - ajuste da locomoção   &100\% &  Configurações câmera V2 &100\% \\ \hline
                Simulação - identificação da tag  &60\% &  Testes odometria         &90\% \\ \hline
                Simulação - ajuste da odometria   &80\% &  Portfólio - post 4 &10\% \\ \hline
            \end{tabular}
        \end{table}
%*----------- notes
    \note[item]{Notes can help you to remember important information. Turn on the notes option.}
\end{frame}
%-
%*----------- SLIDE -------------------------------------------------------------
\begin{frame}[t]{Introdução} 
    \framesubtitle{Planejador global}
    O planejador de trajetória global tem como objetivo \emph{encontrar um caminho} entre duas localizações em um \emph{mapa global}.

    Planejadores globais aderentes ao nav core::BaseGlobalPlanner:
    %\newline
        \begin{columns}[t]
            \column{.05\linewidth}
            \column{.4\linewidth}
                \begin{enumerate}
                    \item Global planner
                    \item Navfn
                    \item Carrot planner
                \end{enumerate}
            \column{.6\linewidth}
            \begin{center}
            %\centerline{
                \begin{figure}
                    %\includegraphics[width=1\textwidth]{pista}
                    % \caption{Pista de corrida \cite{agostini2007}}
                    \roundpic[xshift=-0.7cm,yshift=0.5cm]{2.5cm}{6cm}{trajetoria}
                    %\caption{Pista de corrida \cite{agostini2007}}
                \end{figure}
            %}
            \end{center}
        \end{columns}
        \vspace*{0.6cm}
        Há a possibilidade de inserir plugins com outros métodos de planejamento global.
%*----------- notes
    \note[item]{Notes can help you to remember important information. Turn on the notes option.}
\end{frame}
%-
%*----------- SLIDE -------------------------------------------------------------
\begin{frame}[t]{D*}
    \framesubtitle{Classificações}
    % \begin{columns}
    %     \column{.1\textwidth}
    %     \column{.4\textwidth}
    %     \column{.4\textwidth}
    % \end{columns}

    \begin{block}{D* original}
        -Nome baseado no termo \emph{"Dinâmico A*"}\\
        -Desenvolvido por \emph{Anthony Stentz}
    \end{block}

    \begin{alertblock}{D* focalizado}
        -Combina ideias do \emph{A*} e do \emph{D* original}\\
        -Desenvolvido por \emph{Anthony Stentz}
    \end{alertblock}

    \begin{exampleblock}{D* lite}
        -Baseado no \emph{Lifelong Planning A*}\\
        -Desenvolvido por \emph{Sven Koenig} e \emph{Maxim Likhachev} 
    \end{exampleblock}
 %*----------- notes
    \note[item]{Notes can help you to remember important information. Turn on the notes option.}
\end{frame}
%-
%*----------- SLIDE -------------------------------------------------------------
\begin{frame}[t]{D*}
    \framesubtitle{Funcionamento}
    \begin{enumerate}
        \item Os algoritmos de busca D* resolvem os problemas de planejamento baseados em suposições
        \item Novas informações são adicionadas ao mapa e se necessário uma nova trajetória é planejada
        \item Os sistemas atuais são normalmente baseados no D* lite
    \end{enumerate}

    \begin{figure}
        \includegraphics[trim = 0 20 0 50, clip, width=0.8\textwidth]{funcionamento.png}
        %\caption{.}
    \end{figure}
%*----------- notes
    \note[item]{Notes can help you to remember important information. Turn on the notes option.}
\end{frame}
%-
%*----------- SLIDE -------------------------------------------------------------
\begin{frame}[c]{Metodologia}
    \begin{figure}
        \includegraphics[width=0.8\textwidth]{metodologia.png}
        %\caption{.}
    \end{figure}
%*----------- notes
    \note[item]{Notes can help you to remember important information. Turn on the notes option.}
\end{frame}
%-
%*----------- SLIDE -------------------------------------------------------------
\begin{frame}[c]{Considerações Finais}
    \begin{columns}
        \column{.00\textwidth}
        \column{.5\textwidth}
            \begin{itemize}
                \item Gera uma trajetória global
                \item Dificuldade para \emph{gerar} a trajetória
                \item Dificuldade para \emph{atualizar} a trajetória
                \item \emph{Bons resultados} em curtas distâncias
            \end{itemize}
        \column{.5\textwidth}
             \includegraphics[width=\textwidth]{result.png}
    \end{columns}
 %*----------- notes
    \note[item]{Notes can help you to remember important information. Turn on the notes option.}
\end{frame}
%-